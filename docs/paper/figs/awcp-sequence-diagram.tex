%%%%%%%%%%%%%%%%%%%%%%%%%%%%%%%%%%%%%%%%%%%%%%%%%%%%%%%%%%%%%%%%%%%%%%%%%%%%%%%
% AWCP Protocol Sequence Diagram - Professional UML 2.0 Style
% 
% Usage: %%%%%%%%%%%%%%%%%%%%%%%%%%%%%%%%%%%%%%%%%%%%%%%%%%%%%%%%%%%%%%%%%%%%%%%%%%%%%%%
% AWCP Protocol Sequence Diagram - Professional UML 2.0 Style
% 
% Required: \usepackage{tikz}
%           \usetikzlibrary{positioning, arrows.meta, fit, backgrounds, calc}
%%%%%%%%%%%%%%%%%%%%%%%%%%%%%%%%%%%%%%%%%%%%%%%%%%%%%%%%%%%%%%%%%%%%%%%%%%%%%%%

\begin{figure}[!htb]
\centering
\resizebox{0.6\textwidth}{!}{%
\begin{tikzpicture}[
    % === Core styles ===
    participant/.style={rectangle, draw=black, thick, fill=white,
        minimum width=2.2cm, minimum height=0.7cm, font=\small\bfseries},
    lifeline/.style={dashed, gray!60},
    % === Message arrows ===
    msg/.style={-{Stealth[length=2mm, width=1.5mm]}, thick},
    ret/.style={-{Stealth[length=2mm, width=1.5mm]}, thick, dashed},
    sse/.style={-{Stealth[length=2mm, width=1.5mm]}, thick, blue!70},
    % === Labels ===
    mlabel/.style={font=\scriptsize\ttfamily, fill=white, inner sep=1.5pt},
    plabel/.style={font=\scriptsize\itshape, text=black!70},
    % === Fragment (UML combined fragment) ===
    fragment/.style={draw=black!60, fill=white},
    fraglabel/.style={font=\scriptsize\bfseries, fill=black!10, draw=black!60, 
        inner sep=2pt, anchor=north west}
]

% ========== Participants ==========
\node[participant] (D) at (0, 0) {Delegator};
\node[participant] (E) at (6, 0) {Executor};

% ========== Lifelines ==========
\draw[lifeline] (0, -0.35) -- (0, -13.02);
\draw[lifeline] (6, -0.35) -- (6, -13.02);

%==========================================================================
% PHASE 1: Negotiation
%==========================================================================
\begin{scope}[on background layer]
  \draw[fragment] (-1.2, -0.6) rectangle (7.2, -2.8);
\end{scope}
\node[fraglabel] at (-1.2, -0.6) {Negotiation};

% INVITE
\draw[msg] (0, -1.2) -- node[mlabel, above] {INVITE} (6, -1.2);
\node[plabel] at (3, -1.5) {task, lease, environment};

% ACCEPT (response)
\draw[ret] (6, -2.2) -- node[mlabel, above] {ACCEPT} (0, -2.2);
\node[plabel] at (3, -2.5) {workDir, constraints};

%==========================================================================
% PHASE 2: Provisioning  
%==========================================================================
\begin{scope}[on background layer]
  \draw[fragment] (-1.2, -3.0) rectangle (7.2, -5.8);
\end{scope}
\node[fraglabel] at (-1.2, -3.0) {Provisioning};

% Self-call: Prepare Transport (proper UML self-message)
\draw[msg] (0, -3.6) -- ++(0.8, 0) -- ++(0, -0.3) -- ++(-0.8, 0);
\node[mlabel, anchor=west] at (0.9, -3.7) {Prepare Transport};

% START
\draw[msg] (0, -4.5) -- node[mlabel, above] {START} (6, -4.5);
\node[plabel] at (3, -4.8) {lease, workDir (credentials)};

% {ok} response
\draw[ret] (6, -5.5) -- node[mlabel, above] {\{ok: true\}} (0, -5.5);

%==========================================================================
% PHASE 3: Execution (SSE Stream)
%==========================================================================
\begin{scope}[on background layer]
  \draw[fragment] (-1.2, -6.0) rectangle (7.2, -11.1);
\end{scope}
\node[fraglabel] at (-1.2, -6.0) {Execution};

% Self-call: Setup Workspace
\draw[msg] (6, -6.3) -- ++(-0.8, 0) -- ++(0, -0.3) -- ++(0.8, 0);
\node[mlabel, anchor=east] at (5.1, -6.4) {Setup Workspace};

% SSE subscription
\draw[sse] (0, -7.2) -- node[mlabel, above] {SSE: /tasks/:taskId/events} (6, -7.2);

% status event
\draw[sse] (6, -7.9) -- node[mlabel, above] {event: status} (0, -7.9);
\node[plabel] at (3, -8.2) {running};

% Self-call: Execute Task
\draw[msg] (6, -8.7) -- ++(-0.8, 0) -- ++(0, -0.3) -- ++(0.8, 0);
\node[mlabel, anchor=east] at (5.1, -8.8) {Execute Task};

% Self-call: Teardown
\draw[msg] (6, -9.5) -- ++(-0.8, 0) -- ++(0, -0.3) -- ++(0.8, 0);
\node[mlabel, anchor=east] at (5.1, -9.6) {Teardown};

% snapshot event
\draw[sse] (6, -10.2) -- node[mlabel, above] {event: snapshot} (0, -10.2);
\node[plabel] at (3, -10.5) {snapshotId, data, recommended};

% done event
\draw[sse] (6, -10.9) -- node[mlabel, above] {event: done} (0, -10.9);

%==========================================================================
% PHASE 4: Completion
%==========================================================================
\begin{scope}[on background layer]
  \draw[fragment] (-1.2, -11.3) rectangle (7.2, -13.0);
\end{scope}
\node[fraglabel] at (-1.2, -11.3) {Completion};

% Self-call: Apply Snapshot
\draw[msg] (0, -11.9) -- ++(0.8, 0) -- ++(0, -0.3) -- ++(-0.8, 0);
\node[mlabel, anchor=west] at (0.9, -12.0) {Apply Snapshot};

% ACK
\draw[msg] (0, -12.7) -- node[mlabel, above] {ACK} (6, -12.7);

\end{tikzpicture}%
}
\caption{AWCP four-phase message sequence between Delegator and Executor.
Solid, dashed, and blue arrows denote synchronous HTTP requests, responses, and asynchronous SSE events, respectively.}
\label{fig:awcp-protocol}
\end{figure}

% Required: \usepackage{tikz}
%           \usetikzlibrary{positioning, arrows.meta, fit, backgrounds, calc}
%%%%%%%%%%%%%%%%%%%%%%%%%%%%%%%%%%%%%%%%%%%%%%%%%%%%%%%%%%%%%%%%%%%%%%%%%%%%%%%

\begin{figure}[!htb]
\centering
\resizebox{\columnwidth}{!}{%
\begin{tikzpicture}[
    % === Core styles ===
    participant/.style={rectangle, draw=black, thick, fill=white,
        minimum width=2.2cm, minimum height=0.7cm, font=\small\bfseries},
    lifeline/.style={dashed, gray!60},
    % === Message arrows ===
    msg/.style={-{Stealth[length=2mm, width=1.5mm]}, thick},
    ret/.style={-{Stealth[length=2mm, width=1.5mm]}, thick, dashed},
    sse/.style={-{Stealth[length=2mm, width=1.5mm]}, thick, blue!70},
    % === Labels ===
    mlabel/.style={font=\scriptsize\ttfamily, fill=white, inner sep=1.5pt},
    plabel/.style={font=\scriptsize\itshape, text=black!70},
    % === Fragment (UML combined fragment) ===
    fragment/.style={draw=black!60, fill=white},
    fraglabel/.style={font=\scriptsize\bfseries, fill=black!10, draw=black!60, 
        inner sep=2pt, anchor=north west}
]

% ========== Participants ==========
\node[participant] (D) at (0, 0) {Delegator};
\node[participant] (E) at (6, 0) {Executor};

% ========== Lifelines ==========
\draw[lifeline] (0, -0.35) -- (0, -14.5);
\draw[lifeline] (6, -0.35) -- (6, -14.5);

%==========================================================================
% PHASE 1: Negotiation
%==========================================================================
\begin{scope}[on background layer]
  \draw[fragment] (-1.2, -0.6) rectangle (7.2, -2.8);
\end{scope}
\node[fraglabel] at (-1.2, -0.6) {Negotiation};

% INVITE
\draw[msg] (0, -1.2) -- node[mlabel, above] {INVITE} (6, -1.2);
\node[plabel] at (3, -1.5) {task, lease, environment};

% ACCEPT (response)
\draw[ret] (6, -2.2) -- node[mlabel, above] {ACCEPT} (0, -2.2);
\node[plabel] at (3, -2.5) {workDir, constraints};

%==========================================================================
% PHASE 2: Provisioning  
%==========================================================================
\begin{scope}[on background layer]
  \draw[fragment] (-1.2, -3.0) rectangle (7.2, -6.0);
\end{scope}
\node[fraglabel] at (-1.2, -3.0) {Provisioning};

% Self-call: Prepare Transport (proper UML self-message)
\draw[msg] (0, -3.6) -- ++(0.8, 0) -- ++(0, -0.5) -- ++(-0.8, 0);
\node[mlabel, anchor=west] at (0.9, -3.6) {Prepare Transport};

% START
\draw[msg] (0, -4.5) -- node[mlabel, above] {START} (6, -4.5);
\node[plabel] at (3, -4.8) {lease, workDir (credentials)};

% {ok} response
\draw[ret] (6, -5.5) -- node[mlabel, above] {\{ok: true\}} (0, -5.5);

%==========================================================================
% PHASE 3: Execution (SSE Stream)
%==========================================================================
\begin{scope}[on background layer]
  \draw[fragment] (-1.2, -6.2) rectangle (7.2, -12.0);
\end{scope}
\node[fraglabel] at (-1.2, -6.2) {Execution};

% Self-call: Setup Workspace
\draw[msg] (6, -6.8) -- ++(-0.8, 0) -- ++(0, -0.5) -- ++(0.8, 0);
\node[mlabel, anchor=east] at (5.1, -6.8) {Setup Workspace};

% SSE subscription
\draw[sse] (0, -7.7) -- node[mlabel, above] {SSE: /tasks/:taskId/events} (6, -7.7);

% status event
\draw[sse] (6, -8.4) -- node[mlabel, above] {event: status} (0, -8.4);
\node[plabel] at (3, -8.7) {running};

% Self-call: Execute Task
\draw[msg] (6, -9.2) -- ++(-0.8, 0) -- ++(0, -0.5) -- ++(0.8, 0);
\node[mlabel, anchor=east] at (5.1, -9.2) {Execute Task};

% Self-call: Teardown
\draw[msg] (6, -10.0) -- ++(-0.8, 0) -- ++(0, -0.4) -- ++(0.8, 0);
\node[mlabel, anchor=east] at (5.1, -10.0) {Teardown};

% snapshot event
\draw[sse] (6, -10.7) -- node[mlabel, above] {event: snapshot} (0, -10.7);
\node[plabel] at (3, -11.0) {snapshotId, data, recommended};

% done event
\draw[sse] (6, -11.6) -- node[mlabel, above] {event: done} (0, -11.6);

%==========================================================================
% PHASE 4: Completion
%==========================================================================
\begin{scope}[on background layer]
  \draw[fragment] (-1.2, -12.2) rectangle (7.2, -14.3);
\end{scope}
\node[fraglabel] at (-1.2, -12.2) {Completion};

% Self-call: Apply Snapshot
\draw[msg] (0, -12.8) -- ++(0.8, 0) -- ++(0, -0.5) -- ++(-0.8, 0);
\node[mlabel, anchor=west] at (0.9, -12.8) {Apply Snapshot};

% ACK
\draw[msg] (0, -13.8) -- node[mlabel, above] {ACK} (6, -13.8);

\end{tikzpicture}%
}
\caption{AWCP protocol message sequence. The protocol operates in four phases: 
(1)~\textit{Negotiation}: the Delegator proposes task requirements via \texttt{INVITE}, and the Executor responds with its capabilities and constraints via \texttt{ACCEPT};
(2)~\textit{Provisioning}: the Delegator prepares transport credentials and sends the \texttt{START} message to provision the data plane;
(3)~\textit{Execution}: the Executor sets up the workspace, executes the task, and streams progress and results via Server-Sent Events (SSE);
(4)~\textit{Completion}: the Delegator applies the received snapshot to its local workspace and sends an acknowledgment.
Solid arrows denote synchronous HTTP requests; dashed arrows denote responses; blue arrows denote asynchronous SSE events.}
\label{fig:awcp-protocol}
\end{figure}
